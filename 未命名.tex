%!TEX encoding = UTF-8 Unicode  
\documentclass{article}  
\usepackage{CJKutf8}  

  
\begin{document}  

\begin{CJK}{UTF8}{bsmi}  

\title{稀疏矩阵}
\date{}

\maketitle
就像在上一节描述的一样,标准的离散化的偏微分方程往往会伴随着一个庞大的且稀疏的矩阵。稀疏矩阵可以被模糊的描述为一个具有非常少的非零元的矩阵。但是,事实上,当特殊的技巧需要利用到大量的非零元以及它们的位置时,一个矩阵是可以被稀疏化的。这些稀疏化矩阵的技巧是从不储存零元的想法开始的。一个关键的问题是制定能够适合于高效地使用不论是直接还是迭代的标准计算方法的存储稀疏矩阵的数据结构。这一章节将简介稀疏矩阵,它们的属性、呈现,以及用以存储它们的数据结构。
\newline\newline
3.1介绍
\newline
利用一个矩阵中的零元以及它们的位置的自然的想法最初是由在不同学科的工程师们提出的。在涉及带状矩阵的简单地例子中,特殊的技巧直接的被发明了。在20世纪60年代研发电子网络的电子工程师们是最早的去利用稀疏性来对于具有特殊结构的矩阵解决一般稀疏线性系统。对于稀疏矩阵技巧而言,最主要也是最早需要解决的问题是去设计一个在线性系统中得直接求解算法。这些算法需要是可以接受的,在存储和计算效率上。直接的稀疏算法可以被用于计算那些庞大的难以被稠密算法来实现的问题。


\end{CJK}  
  
\end{document}  
