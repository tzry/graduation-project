%!TEX encoding = UTF-8 Unicode  
\documentclass{article}  
\usepackage{xeCJK}
\setCJKmainfont[BoldFont=STZhongsong, ItalicFont=STKaiti]{STSong}
\setCJKsansfont[BoldFont=STHeiti]{STXihei}
\setCJKmonofont{STFangsong}
\usepackage{graphicx}
\usepackage{amsmath}
 \usepackage{clrscode}
\usepackage{listings}
\usepackage{enumerate}
\lstset{languagev=C++}%这条命令可以让LaTeX排版时将C++键字突出显示

\lstset{breaklines}%这条命令可以让LaTeX自动将长的代码行换行排版

\lstset{extendedchars=false}%这一条命令可以解决代码跨页时,章节标题,页眉等汉字不显示的问题




\begin{document}  

\title{毕业论文}
\date{}

\maketitle


\textbf{中文摘要}关键词
      \qquad
\newline
\textbf{Abstract}Key words
      \qquad
\newline
\textbf{目录}
      \qquad
\newline

\textbf{一、引言}
      \qquad
\newline

\textbf{二、研究背景综述}
      \qquad
\newline

近年来,在工程应用中,求解高阶矩阵的需求日益增长,全矩阵运算脱离了实际的硬件限制,为了满足这一日益增长的需求,同时这些矩阵通常都有着一个特征——非零元远少于零元,稀疏矩阵这门学科便应运而生。
在 20 世纪 60 年代研发电子网络的电子工程师们是最早的去利用稀疏性来应用稀疏矩阵进行工程上的计算的。[1]
而在微分方程数值解、线性规划等的有限元分析中,经常出现求解高阶稀疏线性方程组,如利用全矩阵进行存储,则需要$n^2$的空间复杂度和$n^3$的乘法运算时间复杂度,显然,这种程度的运算量是无法被微型计算机,甚至是工作站所接受的。
而利用矩阵的稀疏性,可以有效地减小消耗很多无谓的存储空间以及无谓的计算,在很大的程度上降低了时间和空间复杂度,降低了计算对硬件的需求,使计算成为可能。
\newline

\textbf{(1)稀疏矩阵的定义}
      \qquad
\newline
稀疏矩阵是非零元远小于矩阵元素总数,且非零元分布没有规律的矩阵。[1]

\textbf{(2)稀疏矩阵已有存储方式}
      \qquad
\newline 
在这些年的发展中,出现了很多的存储方法,比如:对角线存贮法、对称矩阵的变带宽存贮法、坐标存贮法、Elipack-Itpack存贮法、CSR存贮法、Shermans存贮法、超矩阵存贮法、动态存贮方案等[2]。
\newline
\textcircled{1}三元组(Triplet)存储:
顾名思义,三元组存储方法就是分别存储矩阵的非零元所在的行列索引值,以及与之对应的非零元的值的存储方案。[3]
\newline
三元组存储方案有着直观、易于实现、顺序无关的特性,但是与传统的存储方法一样,三元组存储方案不便于求解算法的使用,不利于利用矩阵的稀疏性。
典型的C语言实现为\newline
\begin{lstlisting}

struct triplet_matrix{ 
	int *Ti;/*row pointer*/ 
	int *Tj;/*column pointer*/
	double *Tx;/*value pointer*/
	int Tnz;/*number of entries*/
	int Tnrow;/*number of rows*/
	int Tncol;/*number of columns*/
};

\end{lstlisting}
下面对应于矩阵的三元组存储:
\newline\newline\newline\newline\newline
\includegraphics[scale=0.25]{triplet.png}

\textcircled{2}列压缩CSS(Compressed-column storage)存储:
任何矩阵的运算都涉及到矩阵的存储与读取操作,因此,存取效率对于矩阵运算的效率有着极大的影响。\newline
许多开源库采用了该存储结构,如UMFPACK、TAUCS等。列压缩存储方案较前面的三元组存储方案较不好理解,但是在矩阵计算中能更为高效地利用矩阵的稀疏性,而这也就是各大求解器采用这一存储方案的原因。[3]
\newline
列压缩存储维护了三组数据——(各列非零元的累加值,按递增的顺序存储的每列的非零元的行索引值,对应第二组数据行索引值位置对应的非零元的值)。
典型的C语言结构实现为\newline
\begin{lstlisting}

struct cc_matrix{ 
	int *Ai;/*row index*/ 
	int *Ap;/*length ncol+1*/
	double *Ax;
	
	int Ancol;
	int Anrow;
};

\end{lstlisting}
下面对应于矩阵的列压缩存储:
\newline\newline\newline\newline\newline
\includegraphics[scale=0.25]{ccmatrix.png}

\textcircled{3}行压缩存储方式(Compressed Row Storage):
CRS存储可以高效地存取任意一行非零元素,但存取任意一列非零元则需要遍历整个CRS存储结构。相应地,与CRS存储的稀疏矩阵相关的算法要高效的编程实现,算法的计算顺序必须按行来进行。[3]
下面对应于CRS存储:
\newline\newline\newline\newline\newline

\includegraphics[scale=0.25]{crs.png}
我们可以发现,ROWP数组存储的时行非零元的增长量。COLIND则存储的是列索引值,典型的C语言结构实现为\newline

\begin{lstlisting}

struct cr_matrix{ 
	int *Ri;/*col index*/ 
	int *Rp;/*length nrow+1*/ 
	double *Rx; 
	int Rncol;
	int Rnrow;
};

\end{lstlisting}

显然,越是简单的数据结构,存取操作的效率越高,亦即寻址操作的效率越高。例如,传统的稠密矩阵的存储方案只需根据行列索引值即可计算得到数据所在的地址,轻松高效的实现存取操作。在列压缩存储的稀疏矩阵中,显然能高效地存取矩阵的一列,但是,存取行的效率是极为低下的。因此,在矩阵的运算中,我们希望能尽可能地按列进行,而不是按行进行。


\textbf{(3)已有的处理稀疏矩阵存储与运算的开源库}
      \qquad
\newline
在做稀疏矩阵的计算时,通常都是做一系列的基本运算,如:矩阵转置、矩阵向量乘法、矩阵矩阵乘法、数乘等。为了能得到更好的效率,许多研究者致力于寻找对于这些计算最优的存储结构及计算算法,同时提供了许多类库供科学计算使用,如:Portable, ExtensibleToolkit for Scientific Computation(PETSc)、Boost、GNU Scientific Library (GSL)等。\newline
例如,在Boost-uBLAS中有着稀疏矩阵的模板mapped\_matrix<T, F, A>(元素映射矩阵存储形式)、compressed\_matrix<T, F, IB, IA, TA>(压缩存储格式)、coordinat\_matrix<T, F, IB, IA, TA>(坐标存储格式)。
分别有着如下示例:[4]\newline
\textbf{mapped\_matrix:}
\begin{lstlisting}

#include <boost/numeric/ublas/matrix_sparse.hpp>
#include <boost/numeric/ublas/io.hpp>

int main () {
    using namespace boost::numeric::ublas;
    mapped_matrix<double> m (3, 3, 3 * 3);
    for (unsigned i = 0; i < m.size1 (); ++ i)
        for (unsigned j = 0; j < m.size2 (); ++ j)
            m (i, j) = 3 * i + j;
    std::cout << m << std::endl;
}

\end{lstlisting}

\textbf{compressed\_matrix:}
\begin{lstlisting}

#include <boost/numeric/ublas/matrix_sparse.hpp>
#include <boost/numeric/ublas/io.hpp>

int main () {
    using namespace boost::numeric::ublas;
    compressed_matrix<double> m (3, 3, 3 * 3);
    for (unsigned i = 0; i < m.size1 (); ++ i)
        for (unsigned j = 0; j < m.size2 (); ++ j)
            m (i, j) = 3 * i + j;
    std::cout << m << std::endl;
}

\end{lstlisting}

\textbf{coordinate\_matrix:}
\begin{lstlisting}

#include <boost/numeric/ublas/matrix_sparse.hpp>
#include <boost/numeric/ublas/io.hpp>

int main () {
    using namespace boost::numeric::ublas;
    coordinate_matrix<double> m (3, 3, 3 * 3);
    for (unsigned i = 0; i < m.size1 (); ++ i)
        for (unsigned j = 0; j < m.size2 (); ++ j)
            m (i, j) = 3 * i + j;
    std::cout << m << std::endl;
}
\end{lstlisting}

\textbf{(4)稀疏矩阵的数乘}
      \qquad
\newline
根据数乘的定义,若矩阵A的=\{$a_{ij}$\},则kA=\{$ka_{ij}$\},则稀疏矩阵的数乘可以简单的定义为所有非零元乘上这个数。例如:对(1)实现的三元组存储的稀疏矩阵的数乘可以用如下函数进行计算:
\begin{lstlisting}
void multi(triplet_matrix matrix,double k){
	for(int i=0;i<Tnz;i++){
		*Tx[i]=*Tx[i]*k;
	}
}
\end{lstlisting}

\textbf{(5)稀疏矩阵的矩阵向量乘法}
      \qquad
\newline
y


\textbf{三、研究的目标、内容及研究方法}
      \qquad
\newline
\textbf{(1)研究的目标}
      \qquad
\newline
利用c++在g++编译器linux平台下实现稀疏矩阵的行压缩存储方式(Compressed Row Storage)存储。可以并行计算稀疏矩阵的矩阵乘法、数乘等运算。
\newline

\textbf{(2)研究内容}
      \qquad
\newline
 在linux平台下,基于g++编译器,编写稀疏矩阵存储的C++库,实现稀疏矩阵的行压缩存储方式(Compressed Row Storage),同时实现一些稀疏矩阵的基本运算,如矩阵乘法、数乘等。为了更好地发挥CPU性能,需要实现多线程并行计算。

\textbf{(3)研究方法}
      \qquad
\newline
实现稀疏矩阵的行压缩存储方式(Compressed Row Storage),通过数组存储非零元的增长量、列索引值和非零元值来实现稀疏矩阵的存储。为了实现可变长数组的存储,利用C++的STL中的vector来存储数据。
\newline
而为了更好地利用CPU的性能,利用thread库实现多线程并行计算,更为有效地利用CPU的并行性能。而为了防止多线程并行计算中“脏数据”的出现,利用mutex互斥锁保障进程安全性。
\newline
为了实现稀疏矩阵的矩阵矩阵乘法和矩阵向量乘法,可以参考全矩阵矩阵矩阵乘法和矩阵向量乘法算法,利用稀疏矩阵的零元不参与计算的特性,按照稀疏矩阵存储的结构,较全矩阵计算省略大量计算,实现稀疏矩阵计算的优势——更为高效的计算。
\newline


\textbf{四、实验结果}
      \qquad
\newline
在iMac 5k平台上进行测试(3.5 GHz Intel Core i5,8 GB 1600 MHz DDR3),测试结果如下:

\begin{tabular}{|c|c|c|c|}
\hline \multicolumn{4}{|c|}{10000阶稀疏度为0.01的稀疏矩阵的矩阵矩阵乘法}\\
\hline 单行最大非零元个数&200&500&1000\\
\hline sparseMatrix&85404.1ms&85129.4ms&87244.9ms\\
\hline boost$\_$ublas&43823.6ms&330923ms&1.470550e+06ms\\
\hline
\end{tabular}

\textbf{五、结论}
      \qquad
\newline
	根据本文第四部分内容,boost$\_$ublas库的效率在很大程度上取决于输入参数的好坏,而sparseMatrix则效率与输入参数的好坏相关系数很小。虽然在好的参数条件下,boost$\_$ublas的效率更高,但是,sparseMatrix的鲁棒性更好,不容易受到差的输入参数的影响。



\textbf{参考文献}
      \qquad
\newline
 [1]Yousef Saad.Iterative Methods for Sparse Linear Systems[M].SECOND EDITION.USA:Society for Industrial and Applied Mathematics,2003年.68.\newline
 [2]张永杰、孙秦.稀疏矩阵存储技术[J].长春理工大学学报,2006年,03期:38-41.\newline
 [3]冯广祥. 大型稀疏矩阵直接求解算法的研究及实现[8].  东北大学:系统工程,2010.\newline
 [4]Joerg Walter and Mathias Koch.Boost 1.57.0 Library Documentation-uBLAS. http://www.boost.org/doc/libs/1\_57\_0/libs/numeric/ublas/doc/index.html, 2011\newline

  \end{document}  
