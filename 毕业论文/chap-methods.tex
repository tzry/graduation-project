%!TEX encoding = UTF-8 Unicode  
\chapter{研究的目标、内容及研究方法}
\section{研究的目标}
利用c++语言在linux平台下的g++编译器上编写一个库,实现稀疏矩阵的行压缩存储方式(Compressed Row Storage)存储,并可以实现并行计算稀疏矩阵的矩阵乘法、数乘等运算,可以在实际的科学计算中进行使用。
 \section{研究内容}
  在linux平台下,基于g++编译器、C++标准库,编写稀疏矩阵存储的C++库,实现稀疏矩阵的行压缩存储方式(Compressed Row Storage),同时实现一些稀疏矩阵的基本运算,如矩阵乘法、数乘等。而为了更好地发挥现代多核心CPU的性能,需要实现多线程并行计算,为以后可能进行的集群运算留下可以使用的接口。
  
  \section{研究方法}
  实现稀疏矩阵的行压缩存储方式(Compressed Row Storage),通过数组存储非零元的增长量、列索引值和非零元值来实现稀疏矩阵的存储。为了实现可变长数组的存储,利用C++的STL中的vector来存储数据。
  
  具体实现数据结构如下:
  \begin{lstlisting}

class sparseMatrix
{
private:
    int n_col;		/**< number of the columns. */
    int n_row;		/**< number of the rows. */
    int n_couple;
public:
    std::vector<double> ele; /**< the array to story the values of the elements. */
    std::vector<int> col;	/**< the column  index of the non-zero elements. */
    std::vector<int> row;	/**< the row index of the non-zero elements. */
}
\end{lstlisting}

而为了更好地利用现代多核心CPU的性能,利用thread库实现多线程并行计算,更为有效地利用CPU的并行性能,为未来用于集群计算留下可以使用的接口。而在并行计算中,由于存在多个线程同时操作某一块内存,出现“脏数据”的情况,为了防止多线程并行计算中“脏数据”的出现,利用mutex互斥锁保障进程安全性,将正在运算的内存进行加锁,防止被另一线程同时操作,在操作完成后,进行解锁,供另一线程操作。而在不同的机型上,由于硬件配置、软件配置不一,最优线程数也不一,为了实现更好的利用计算机硬件性能,利用C++的预定义来由程序编写者自行设定线程数。
\newline
由于在实际工程中,经常遇到的矩阵操作有矩阵数乘运算、矩阵矩阵乘法、矩阵向量乘法等等,需要实现稀疏矩阵的一些基本运算(矩阵数乘运算、矩阵矩阵乘法、矩阵向量乘法等)。而为了实现稀疏矩阵的矩阵矩阵乘法和矩阵向量乘法,可以参考全矩阵矩阵矩阵乘法和矩阵向量乘法算法,利用稀疏矩阵的零元不参与计算的特性,按照稀疏矩阵存储的结构,较全矩阵计算省略大量计算,实现稀疏矩阵计算的优势——更为高效的计算。

  \chapter{实验结果}
  在iMac 5k平台上进行测试(3.5 GHz Intel Core i5,8 GB 1600 MHz DDR3),分别测试sparseMatrix和boost-ublas,测试代码如下:
  
  \textbf{使用sparseMatrix.h}
  \begin{lstlisting}

#include <iostream>
#include "sparseMatrix.h"
#include<stdlib.h>
#include "complexSparseMatrix.h"

#define ROW 100000 //行数
#define COL 100000 //列数
#define COUPLE 1500   //每行最多非零数
#define NONE_ZERO_PERCENT 0.01


int main(int argc, const char * argv[]) {
    //初始化稀疏矩阵
    sparseMatrix* sparse=new sparseMatrix(ROW,COL,COUPLE);
    sparse->init(NONE_ZERO_PERCENT);
    sparse->compress();
  
    clock_t start,finish;double duration;
    start=clock();
    multi(sparse, sparse);
    finish=clock();
    duration=1000*(double)(finish-start)/CLOCKS_PER_SEC;
    std::cout<<duration<<std::endl;
    
    multi(sparse, sparse)->show('B');
    
    return 0;
}


\end{lstlisting}
\textbf{使用boost\_ublas}
\begin{lstlisting}
  #include <iostream>
#include<fstream>
#include "boost/numeric/ublas/matrix_sparse.hpp"
#include "boost/numeric/ublas/io.hpp"
using namespace std;
int main ( ) {
    using namespace boost::numeric::ublas ;
    compressed_matrix <double> m ( 10000 , 1000) ;
    
    ifstream infile;
    infile.open("/Users/tzry/Desktop/graduation-project/sparse/output.txt");
    
    while (infile.good()&&!infile.eof()) {
        int i,j;
        double v;
        infile>>i>>j>>v;
        m(i-1,j-1)=v;
        
    }
    
    
    clock_t start,finish;double duration;
    start=clock();

    m=boost::numeric::ublas::prod(m,m);
    
    finish=clock();
    duration=1000*(double)(finish-start)/CLOCKS_PER_SEC;
    std::cout<<duration<<std::endl;

}
\end{lstlisting}
 测试结果如下:

\begin{tabular}{|c|c|c|c|}
\hline \multicolumn{4}{|c|}{10000阶稀疏度为0.01的稀疏矩阵的矩阵矩阵乘法}\\
\hline 单行最大非零元个数&200&500&1000\\
\hline sparseMatrix&85404.1ms&85129.4ms&87244.9ms\\
\hline boost$\_$ublas&43823.6ms&330923ms&1.470550e+06ms\\
\hline
\end{tabular}



\chapter{结论}
根据本文第四部分内容,boost$\_$ublas库的效率在很大程度上取决于输入参数的好坏,而sparseMatrix则效率与输入参数的好坏相关系数很小。虽然在好的参数条件下,boost$\_$ublas的效率更高,但是,sparseMatrix的鲁棒性更好,不容易受到差的输入参数的影响。sparseMatrix在较差的参数条件下与较好地参数条件下,效率基本没有区别。
\newline
在实际计算中,遇到不容易知道参数如何选择,或者每行的非零元数量经常变化,且变化较大时,使用sparseMatrix库能有效地提高使用效率。






