%!TEX encoding = UTF-8 Unicode  
\documentclass{article}  
\usepackage{xeCJK}
\setCJKmainfont[BoldFont=STZhongsong, ItalicFont=STKaiti]{STSong}
\setCJKsansfont[BoldFont=STHeiti]{STXihei}
\setCJKmonofont{STFangsong}
\usepackage{graphicx}
\usepackage{amsmath}
 \usepackage{clrscode}



\begin{document}  

\title{文献综述}
\date{}

\maketitle

摘要:
\newline
\textbf{一、背景介绍}
      \qquad

利用一个矩阵中的零元以及它们的位置的自然的想法最初是由在不同学科的工 程师们提出的。在涉及带状矩阵的简单地例子中,特殊的技巧直接的被发明了。 在 20 世纪 60 年代研发电子网络的电子工程师们是最早的去利用稀疏性来对于 具有特殊结构的矩阵解决一般稀疏线性系统。[1]自然,稀疏矩阵的存储问题是一个十分重要的问题,在这些年的发展中,出现了很多的存储方法,比如:对角线存贮法、对称矩阵的变带宽存贮法、坐标存贮法、Elipack-Itpack存贮法、CSR存贮法、Shermans存贮法、超矩阵存贮法、动态存贮方案等[2]。
\newline
二、国内外研究现状\newline
1、研究方向及进展\newline
自稀疏矩阵出现以来,稀疏矩阵的存储问题自然是稀疏矩阵计算中不可避免的一个重要问题。在早期计算机时,串行的算法占到主流,但是随着计算机的发展,计算机集群、多核CPU、GPU并行等的出现,让并行算法在计算时间上远远地超越了串行算法,为了更好的适应并行计算中的稀疏矩阵计算,出现了许多新的算法以及存储方式。例如对于非结构化的矩阵,基于CUDA框架下的SCOO形式的SpMV算法比利用基于Cusp库的SCOO具有更高的效率。[y]
在对称矩阵乘法,或者转置的矩阵向量乘法中,RSB格式的迭代算法效率是比较高的[x]。
2、存在问题\newline

3、研究展望\newline


\textbf{三、参考文献}
      \qquad
\newline
 [1]Yousef Saad.Iterative Methods for Sparse Linear Systems[M].SECOND EDITION.USA:Society for Industrial and Applied Mathematics,2003年.68.\newline
 [2]张永杰、孙秦.稀疏矩阵存储技术[J].长春理工大学学报,2006年,03期:38-41.\newline
  
  [x]Michele Martone.Efficient multithreaded untransposed, transposed or symmetric sparse matrix–vector multiplication with the Recursive Sparse Blocks format[J].Parallel Computing, 2014,40:47-58.\newline
   [y]Hoang-Vu Dang,  Bertil Schmidt.CUDA-enabled Sparse Matrix–Vector Multiplication on GPUs
using atomic operations[J].Parallel Computing, 2013,Vol.39 (11):737-750.\newline
  

\end{document}  
