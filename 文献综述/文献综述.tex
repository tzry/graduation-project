%!TEX encoding = UTF-8 Unicode  
\documentclass{article}  
\usepackage{xeCJK}
\setCJKmainfont[BoldFont=STZhongsong, ItalicFont=STKaiti]{STSong}
\setCJKsansfont[BoldFont=STHeiti]{STXihei}
\setCJKmonofont{STFangsong}
\usepackage{graphicx}
\usepackage{amsmath}
 \usepackage{clrscode}



\begin{document}  

\title{文献综述}
\date{}

\maketitle

摘要:
\newline
\textbf{一、背景介绍}
      \qquad

利用一个矩阵中的零元以及它们的位置的自然的想法最初是由在不同学科的工 程师们提出的。在涉及带状矩阵的简单地例子中,特殊的技巧直接的被发明了。 在 20 世纪 60 年代研发电子网络的电子工程师们是最早的去利用稀疏性来对于 具有特殊结构的矩阵解决一般稀疏线性系统。[1]自然,稀疏矩阵的存储问题是一个十分重要的问题,在这些年的发展中,出现了很多的存储方法,比如:对角线存贮法、对称矩阵的变带宽存贮法、坐标存贮法、Elipack-Itpack存贮法、CSR存贮法、Shermans存贮法、超矩阵存贮法、动态存贮方案等[2]。




\newline
二、国内外研究现状\newline
1、研究方向及进展\newline

2、存在问题\newline

3、研究展望\newline

\newline
\textbf{三、参考文献}
      \qquad
\newline
 [1]Yousef Saad.Iterative Methods for Sparse Linear Systems[M].SECOND EDITION.USA:Society for Industrial and Applied Mathematics,2003年.68.\newline
  [2]张永杰、孙秦.稀疏矩阵存储技术[J].长春理工大学学报,2006年,03期:38-41.\newline

\end{document}  
